\section{Tabellen}

Tabellen werden nur mit horizontalen Balken gesetzt, siehe~\cref{tab:example}.

\begin{table}[htbp]
  \centering
  \begin{tabularx}{\textwidth}{lXlr}\toprule
    \# & Land      & ccTLD      & BIP/Kopf 2019 \\ \midrule
    1  & Luxemburg & \verb|.lu| & 115.839 \$    \\
    2  & Schweiz   & \verb|.ch| & 82.484 \$     \\
    3  & Irland    & \verb|.ie| & 80.504 \$     \\
    4  & Norwegen  & \verb|.no| & 75.294 \$     \\
    5  & Island    & \verb|.is| & 67.857 \$     \\ \bottomrule
  \end{tabularx}
  \caption{Beispieltabelle mit verschiedenen Daten\label{tab:example}}
\end{table}

\section{Mathematik}

Formeln können im laufenden Text verwendet werden, \(a^2 + b^2 = c^2\),
oder als eigener Block, siehe~\cref{eq:triangular}.

\begin{equation}
  \label{eq:triangular}
  T_n \coloneq \sum^{n}_{k=1} = \frac{n(n+1)}{2}
\end{equation}
