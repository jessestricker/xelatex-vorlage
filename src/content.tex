\section{Einheiten}

Es können Einheiten (\unit{\ohm\per\meter}), Zahlen (\num{123456}) und
Größen (\qty{3.14}{\milli\meter}) verwendet werden.

Außerdem gibt es Listen (\numlist{1;2;3;5;7}) und
Intervalle (\qtyrange{0.13}{0.67}{\milli\metre}).

\section{Tabellen}

Tabellen werden nur mit horizontalen Balken gesetzt, siehe~\cref{tab:example}.

\begin{table}[htbp]
  \centering
  \begin{tabularx}{\textwidth}{lXrrr}\toprule
    \# & Land      & ccTLD        & Fläche / \unit{km^2} & BIP/Kopf / \unit{US\$} (2019) \\ \midrule
    1  & Luxemburg & \texttt{.lu} & \num{2586}           & \num{115839}                  \\
    2  & Schweiz   & \texttt{.ch} & \num{41285}          & \num{82484}                   \\
    3  & Irland    & \texttt{.ie} & \num{69797}          & \num{80504}                   \\
    4  & Norwegen  & \texttt{.no} & \num{385207}         & \num{75294}                   \\
    5  & Island    & \texttt{.is} & \num{103125}         & \num{67857}                   \\ \bottomrule
  \end{tabularx}
  \caption{Beispieltabelle mit verschiedenen Daten\label{tab:example}}
\end{table}

\section{Mathematik}

Formeln können im laufenden Text verwendet werden, \(a^2 + b^2 = c^2\),
oder als eigener Block, siehe~\cref{eq:triangular}.

\begin{equation}
  \label{eq:triangular}
  T_n \coloneq \sum^{n}_{k=1} = \frac{n(n+1)}{2}
\end{equation}
