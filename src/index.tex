\documentclass{scrartcl}

\usepackage{mathtools}
\usepackage[math-style=ISO]{unicode-math}
\usepackage{libertinus-otf}

\usepackage{polyglossia}
\setdefaultlanguage{de-DE}
\usepackage[useregional]{datetime2}
\usepackage{cleveref}

\usepackage{metalogo}

\title{\XeLaTeX-Vorlage}
\author{Jesse Stricker}
\date{\DTMdate{2000-01-01}}

\begin{document}

\maketitle

\section{Mathematik}

Formeln können im laufenden Text verwendet werden, \(a^2 + b^2 = c^2\),
oder als eigener Block, siehe~\cref{eq:triangular}.

\begin{equation}
  \label{eq:triangular}
  T_n \coloneqq \sum^{n}_{k=1} = \frac{n(n+1)}{2}
\end{equation}

\end{document}
